\chapter{Group Theory}

\section{Definition of group structure}

\subsection{Article 26. Groups}

\paragraph{Problem 26$\alpha$.}
Show that the set of all mappings from a given set $X$ to itself forms a
semigroup in which the product is composition of mappings. Show that the
set of all one-to-one correspondences of $X$ with itself forms a group
under composition.

\paragraph*{Solution}
Let $F = \{ f \mid f : X \rightarrow X \}$. For $ f, g, h \in F$, we have
$(fg)h = f(gh)$ by \textbf{16}. Thus $F$ is a semigroup.

Let $ G \subset F$ be the set of all one-to-one correspondences. The mapping
$ 1_G \in G $ (where $1_G x = x$ for all $ x \in G$) is the identity element
since $1_G f = f 1_G = f $ for all $ f \in F$ (and thus $G$). Since each
$f \in G$ is a one-to-one correspondence, $ f^{-1} $ exists, $ f^{-1} \in G$ and
$f^{-1} f = f f^{-1} = 1_G$.

\paragraph{Problem 26$\beta$.}
Let $S$ be a semigroup with an element $e$ such that $ea = a = ae$ for all
$a \in S$. Show that $e$ is unique.

\paragraph*{Solution}
Suppose there were another element $e'$ such that $e'a = a = ae'$ for all
$a \in S$. Then $ e = ee' = e'$.

\paragraph{Problem 26$\gamma$.}
Let $S$ be a semigroup with an element $e$ such that $ea = a$ for all $a \in S$
and such that for every $a \in S$ there exists an element $a^{-1}$ for which
$a^{-1} a = e$. Prove that $S$ is a group.

\paragraph*{Solution}
$ ee = e = e^{-1} e $.
$ae = e(ae) = (ea)e = ae $.
$a^{-1} a = (a^{-1} a) (a^{-1} a) = e$.
??

\paragraph{Problem 26$\epsilon$.}
Let $G$ be a group. Define a new product on $G$ by $a * b = ba $ for any
$a, b \in G$. Show that $G^*$ (the set $G$ with product $*$) is a group.

\paragraph*{Solution}
(1) $(a*b)*c = c(ba) = (cb)a = a*(b*c)$ for any $a, b, c \in G$. (2) The identity
$e$ in $G$ is the identity in $G^*$: $e * a = ae = a = ea = a * e$.
(3) Inverses are also the same: $a * a^{-1} = a^{-1} a = e = a a^{-1} = a^{-1} * a$.

\paragraph{Problem 26$\zeta$.}
Let $G$ and $G'$ be groups. Define a product operation on the set $G \times G'$ by the
rule $(a, a')(b, b') = (ab, a'b')$. Show that $G \times G'$ is a group under this
product. ($G \times G'$ is called the \textit{direct product} of $G$ and $G'$).

\paragraph*{Solution}
Since $ab \in G$ and $a'b' \in G'$, we have $(ab, a'b') \in G \times G'$ i.e. the
product is closed. Next we have $((a,a')(b,b'))(c,c') = (ab, a'b')(c,c') =
(abc, a'b'c') = (a, a')(bc, b'c') = (a,a')((b,b')(c,c')) $ so that the product
is associative. The identity is $(e, e')$ since $(a,a')(e,e') = (ae, a'e') =
(a,a') = (ea, e'a') = (e,e')(a,a')$. The inverse of $(a, a')$ is $(a^{-1},(a')^{-1}$
since $(a,a')(a^{-1},(a')^{-1}) = (aa^{-1},a'(a')^{-1}) = (e,e') =
(a^{-1}a, (a')^{-1}a') = (a^{-1},(a')^{-1})(a,a')$.

\paragraph{Problem 26$\eta$.}

\paragraph*{Solution}
There are six elements in the group:
\begin{center}
\begin{tabular}{c|c}
Rotations & Inversions \\
\hline
$ABC \rightarrow ABC$ & $ABC \rightarrow CBA $ \\
$ABC \rightarrow BCA$ & $ABC \rightarrow ACB $ \\
$ABC \rightarrow CAB$ & $ABC \rightarrow BAC $
\end{tabular}
\end{center}

\paragraph{Problem 26$\theta$.}
The group of symmetries of a regular polygon of $n$ sides is called the
\textit{dihedral group $D_n$}. How many elements does $D_n$ have?

\paragraph*{Solution}
The order of the sides and angles cannot be changed, so only two choices
affect the symmetry: where to place an angle, and what side of the polygon
is forward. For the angle there are $n$ choices of corners. For the side
there are $2$ choices. This gives $2n$ elements of the group.

\paragraph{Problem 26$\kappa$.}
Show that the power set $2^X$ of any set $X$ is a group under the operation
of symmetric difference $A * B$ (\textbf{8$\alpha$}).

\paragraph*{Solution}
$A * B \in 2^X $ for any $A, B \in X$. The symmetric difference is associative
(see \textbf{8$\alpha$}). $A*\emptyset = \emptyset*A = A$ so that $\emptyset$
plays the role of identity. Finally, $ A*A = \emptyset $ so each element is its
own inverse.

\paragraph{Problem 26$\lambda$.}
Show that the set $(-1, 1)$ of real numbers $x$ such that $ -1 < x < 1$ forms
a group under the operation $x y = (x +y) / (1 + xy)$.

\paragraph*{Solution}
\begin{eqnarray*}
(xy)z = & \frac{\frac{x+y}{1+xy} + z}{1+\frac{x+y}{1+xy}z} \\
      = & \frac{\frac{x+y}{1+xy} + z}{1+\frac{x+y}{1+xy}z} \frac{1+xy}{1+xy} \\
      = & \frac{x + y + z + xyz}{1 + xy + xz + yz} \\
      = & \frac{x(1+yz) + (y+z)}{(1+yz) + x(y+z)} \\
      = & \frac{x + \frac{y+z}{1+yz}}{1 + x \frac{y+z}{1+yz}} \\
      = & x(yz)
\end{eqnarray*}

$e = 0$: $xe = (x + 0)/(1 + x0) = x = (0 + x)/(1 + 0x) = ex$.

$x^{-1} = -x$: $xx^{-1} = (x + -x)/(1 + x(-x)) = 0 = (-x + x)/(1+(-x)x) = x^{-1}x$.

\paragraph{Problem 26$\nu$.}

\paragraph*{Solution}
Define $(a_1, a_2, \dots, a_n)(b_1,b_2,\dots,b_n) = (a_1 b_1, a_2 b_2, \dots, a_n b_n)$
for elements in $G_1 \times G_2 \times \dots \times G_n$.

\subsection{Article 27.}

\paragraph{Problem 27$\alpha$.}
Show that the five distinct ways of associating a product of four group elements
in a given order are all equal.

\paragraph*{Solution}
Making lots of use of the associativity axiom, we get
$ ((ab)c)d = (a(bc))d = a((bc)d) = a(b(cd)) = (ab)(cd) $.

\subsection{Article 28. Implications of group structure.}

\paragraph{Problem 28$\beta$.}
Let $a$ be an element of a group $G$. Show that the mapping $ \lambda_a : G \rightarrow G$
given by $ \lambda_a g = ag$ for any $g \in G$ is a one-to-one correspondence.

\paragraph*{Solution}
Since $G$ is a group, for any $g \in G$ there exists an element $h= a^{-1} g \in G$ such
that $ \lambda_a h = ah = aa^{-1}g = g$ so that $\lambda_a$ is onto. Suppose
next that $\lambda_a g = \lambda_a h$ for two elements $g, h \in G$. Then
$ ag = ah $ which implies $ g = h $ and thus $\lambda_a $ is one-to-one, hence
a one-to-one correspondence since $\lambda_a$ is also onto.

\paragraph{Problem 28$\gamma$.}
A group $G$ is \textit{isomorphic} to a group $G'$ if there exists a one-to-one
correspondence $\phi : G \rightarrow G$ such that $ \phi(ab) = (\phi a) (\phi b)$,
or in other words, such that $\phi$ preserves group products. Show that such a
mapping $\phi$ (called an \textit{isomorphism}) also preserves identity elements
and inverses.

\paragraph*{Solution}
$\phi$ preserves identity elements because $\phi(e) = \phi(ee) = (\phi e)(\phi e)$
which implies $ \phi e = e'$ ($e'$ is the identity of $G'$). $\phi$ preserves
inverses since $e' = \phi e = \phi (a a^{-1}) = (\phi a)(\phi a^{-1})$ so that
$(\phi a)^{-1} = \phi a^{-1}$ since inverses are unique (\textbf{28}).

\subsection{Article 29. Abelian groups}

\paragraph{Problem 29$\beta$.}
Prove that a group with less than six elements is abelian. Construct a group
with six elements that is not abelian, and one that is.

\paragraph*{Solution}
A group with one element, $G = \left\{ e \right\}$ is obviously abelian. Groups
with 2, 3, and 5 elements all have prime order and are cyclic, hence they are
abelian. There are only two distinct groups with four elements (up to isomorphism).
One of them is cyclic and hence abelian. The other is $D_2$, the symmetries of
a rectangle (see Table \ref{table:group_d2}). By inspection, this is abelian
as well.

\begin{table}[ht]
\begin{center}
\begin{tabular}[ht]{c|cccc}
    & $e$ & $a$ & $b$ & $c$ \\
\hline
$e$ & $e$ & $a$ & $b$ & $c$ \\
$a$ & $a$ & $e$ & $c$ & $b$ \\
$b$ & $b$ & $c$ & $e$ & $a$ \\
$c$ & $c$ & $b$ & $a$ & $e$ \\
\end{tabular}
\end{center}
\caption{The group $D_2$}
\label{table:group_d2}
\end{table}

A group with six elements that is not abelian is $D_3$. A group with six
elements that is abelian is $\mathbb{Z}_6 = \left\{[0]_6, [1]_6, \dots ,[5]_6 \right\}$.

\paragraph{Problem 29$\delta$.}
Show that a group $G$ in which $x^2 = e$ for every $ x \in G$ is an abelian group.

\paragraph*{Solution}
$x^2 = e $ implies $x = x^{-1}$ for every $ x \in G$.
This gives $ ab = (ab)^{-1} = b^{-1} a^{-1} = ba$.


\section{Examples of group structure}


\subsection{Article 30. Symmetric group on n letters}

\paragraph{Problem 30$\alpha$.}
Show that $S_n$ has $n!$ elements.

\paragraph*{Solution}
$S_n$ is the collection of all one-to-one correspondences
$f:\mathbb{N}_k \rightarrow \mathbb{N}_k$. There are $n$ choices for the value
of $f(1)$. Once $f(1)$ is determined, there are $n-1$ choices for the value of
$f(2)$. Continuing this way, we have $n \cdot (n-1) \cdots 2 \cdot 1 = n!$ distinct
one-to-one correspondences.

\paragraph{Problem 30$\beta$.}
Show that $S_n$ is not abelian for $n > 2$.

\paragraph*{Solution}
Consider $S_3$. Let $\pi$ be defined as $\pi(1) = 1$, $\pi(2) = 3$ and $\pi(3) = 2$,
and let $\rho$ be defined as $\rho(1) = 2$, $\rho(2) = 3$ and $\rho(3) = 1$. Clearly
$\pi, \rho \in S_3$. Now we have $\pi \rho (1) = \rho (\pi (1)) = \rho (1) = 2$,
$\pi \rho (2) = 1$ and $\pi \rho (3) = 3$, but $\rho \pi (1) = 3$, $\rho \pi (2) = 2$
and $\rho \pi (3) = 1$. Thus $\pi \rho \neq \rho \pi$ and $S_3$ is not abelian.

Now suppose $S_{n-1}$ is not abelian. Then there exist elements, say $\pi_{n-1}$
and $\rho_{n-1}$ such that $\pi_{n-1} \rho_{n-1} \neq \rho_{n-1} \pi_{n-1}$.
Consider the elements $\pi_n, \rho_n \in S_n$ defined as $\pi_n(k) = \pi_{n-1}(k)$,
$\rho_n(k) = \rho_{n-1}(k)$ for $k = 1, 2, \dots, n-1$, and $\pi_n(n) = \rho_n(n) = n$.
Since $\pi_{n-1} \rho_{n-1} \neq \rho_{n-1} \pi_{n-1}$, there exists some
$k' \in \mathbb{N}_{n-1}$ such that $\pi_{n-1} \rho_{n-1} (k') \neq \rho_{n-1} \pi_{n-1} (k')$.
But then $\pi_n \rho_n (k') \neq \rho_n \pi_n (k')$ since $k' < n$. Thus
$\pi_n \rho_n \neq \rho_n \pi_n$ and $S_n$ is not abelian.

\paragraph{Problem 30$\gamma$.}
Construct a group isomorphism from $S_3$ to the dihedral group $D_3$.

\paragraph*{Solution}
(\textit{Sketch}) Note that rotations in $D_3$ correspond to the permutations
$(1,2,3)$, $(3,1,2)$, and $(2,3,1)$ in $S_3$. Rotations plus a flip in $D_3$
correspond to the permutations $(1,3,2)$, $(2,1,3)$, and $(3,2,1)$. A formal
isomorphism can be constructed by identifying those elements.


\section{Subgroups and cosets}


\subsection{Article 35. Subgroups}

\paragraph{Problem 35$\alpha$.}
Show that a nonempty subset $H$ of a group $G$ is a subgroup of $G$ if and only
if $a, b \in H$ implies $ab^{-1} \in H$.

\paragraph*{Solution}
$a,a \in H$ implies $aa^{-1} = e \in H$, so $e,b \in H$ implies $eb^{-1} = b^{-1}
\in H$. Next, $a, b \in H$ implies $a, b^{-1} \in H$ implies $a(b^{-1})^{-1} =
ab \in H$.

\paragraph{Problem 35$\beta$.}
Let $H_1, H_2, \dots, H_n$ be subgroups of a group $G$. Show that
$H = \cap_{k=1}^n H_k$ is a subgroup.

\paragraph*{Solution}
Since $e \in H_k \mbox{ } \forall k$, $H$ is nonempty.
Suppose $a, b \in H$. Then $a, b \in H_k \mbox{ } \forall k$ which implies
$ab \in H_k \mbox{ } \forall k$ so that $ab \in H$. Similar argument holds for $a^{-1}$.

\paragraph{Problem 35$\gamma$.}
Let $H$ be a subgroup of $G$ and let $a \in G$. Let
$$ H^a = \left\{x \in G \mid axa^{-1} \in H \right\}. $$
Show that $H^a$ is a subgroup of $G$.

Let $$N(H) = \left\{ a \in G \mid H^a = H \right\}.$$
Show that $N(H)$ is a subgroup of $G$ and $H$ is a subgroup of $N(H)$.
($N(H)$ is called the \textit{normalizer} of $H$.)

\paragraph*{Solution}
Since $aea^{-1}=e\in H$, we have $e\in H^a$ so that $H^a$ is not empty.

Suppose $x, y \in H^a$. Then $axa^{-1} \in H$ and $aya^{-1} \in H$ imply that
$(axa^{-1})(aya^{-1}) = a(xy)a^{-1} \in H$ because $H$ is a subgroup. Hence
$xy \in H^a$.

Suppose $x \in H^a$. Then $axa^{-1} \in H$ implies
$(axa^{-1})^{-1} = a(x^{-1})a^{-1} \in H$ since $H$ is a subgroup. Hence
$x^{-1} \in H^a$.

This shows that $H^a$ is a subgroup of $G$.

\paragraph{}
Clearly $N(H) \subset G$. Since $H^e=\left\{x\in G \mid x\in H \right\}=H$,
$e \in N(H)$ and $N(H)$ is not empty.

Suppose $x, y \in N(H)$. Then $H^x = H = H^y$. By definition, this means
$xzx^{-1} \in H$ if and only if $z \in H$ if and only if $yzy^{-1} \in H$.
Then we have $x(yzy^{-1})x^{-1} = (xy)z(xy)^{-1} \in H$ so that $xy \in N(H)$.

If $x \in N(H)$, then $H^x=H$ which means $xzx^{-1} \in H$ if and only if
$z \in H$. Since $H$ is a subgroup, $z \in H \Leftrightarrow z^{-1} \in H$ so that
$xz^{-1}x^{-1} \in H$. Therefore $z \in H$ if and only if
$(xz^{-1}x^{-1})^{-1} = x^{-1}zx \in H$ and $H^{x^{-1}} = H$. Thus $x^{-1} \in N(H)$.

This shows that $N(H)$ is a subgroup of $G$.

\paragraph{}
Since $H$ is a subgroup of $G$, we know $H$ is a group so we need only show
$H \subset N(H)$. Suppose $h \in H$. Then $H^h = \left\{x \in G \mid hxh^{-1} \in H \right\}$.
We have $hxh^{-1} \in H$ if and only if $x \in h^{-1}Hh = H$. Thus $H^h=H$,
$h\in N(H)$, and $H \subset N(H)$.

This shows that $H$ is a subgroup of $N(H)$.

\paragraph{Problem 35$\delta$.}
Let $Z_G$ be the set of all elements of a group $G$ which commute with all the
elements of $G$; that is
$$ Z_G = \left\{ x \in G \mid xa = ax, a\in G \right\}.$$
Show that $Z_G$ is a subgroup of $G$. ($Z_G$ is called the \textit{center} of $G$.)

\paragraph*{Solution}
Suppose $x, y \in Z_G$. Then for any $a\in G$ we have $xa=ax$ and $ya=ay$, and
$xya=xay=axy$. Hence $xy \in Z_G$.

Suppose $x \in Z_G$. Then for any $a\in G$ we have $xa=ax$ and therefore
$x^{-1}a=x^{-1}(ax)x^{-1}=x^{-1}(xa)x^{-1}=ax^{-1}$ so that $x^{-1} \in Z_G$.

This shows that $Z_G$ is a subgroup of $G$.

\paragraph{Example}
As an example of the above material, consider the group $D_4$.

\begin{center}
\begin{tabular}{c||c|c|c|c||c|c|c|c|}
      & $e$ & $r$ & $r^2$ & $r^3$ & $a$ & $b$ & $c$ & $d$ \\
\hline \hline
$e$   & $e$ & $r$ & $r^2$ & $r^3$ & $a$ & $b$ & $c$ & $d$ \\
\hline
$r$   & $r$ & $r^2$ & $r^3$ & $e$ & $d$ & $c$ & $a$ & $b$ \\
\hline
$r^2$ & $r^2$ & $r^3$ & $e$ & $r$ & $b$ & $a$ & $d$ & $c$ \\
\hline
$r^3$ & $r^3$ & $e$ & $r$ & $r^2$ & $c$ & $d$ & $b$ & $a$ \\
\hline \hline
$a$   & $a$ & $c$ & $b$ & $d$ & $e$ & $r^2$ & $r$ & $r^3$ \\
\hline
$b$   & $b$ & $d$ & $a$ & $c$ & $r^2$ & $e$ & $r^3$ & $r$ \\
\hline
$c$   & $c$ & $b$ & $d$ & $a$ & $r^3$ & $r$ & $e$ & $r^2$ \\
\hline
$d$   & $d$ & $a$ & $c$ & $b$ & $r$ & $r^3$ & $r^2$ & $e$ \\
\hline
\end{tabular}
\end{center}

The center $Z_{D_4}$ is $\left\{e, r^2\right\}$.

\subsection{Article 36. Subgroups of $\mathbb{Z}$.}

\paragraph{Problem 36$\alpha$.}
Describe the subgroup $n\mathbb{Z} \cap m\mathbb{Z}$.

\paragraph*{Solution}
$[m,n]\mathbb{Z}$ where $[m,n]$ is the least common multiple.

\paragraph{Problem 36$\beta$.}
Describe the subgroup of $\mathbb{Z}$ generated by $m$ and $n$, that is,
by the subset $\{ m, n\}$.

\paragraph*{Solution}
$\left\{ pn + qm \mid p,q \in \mathbb{Z} \right\} = (m,n)\mathbb{Z}$ where
$(m,n)$ is the greatest common divisor.

\paragraph{Problem 36$\delta$.}
Show that $\mathbb{Z} \times \mathbb{Z}$ has subgroups not of the form
$n\mathbb{Z} \times m\mathbb{Z}$.

\paragraph*{Solution}
Each of the following are subgroups of $\mathbb{Z} \times \mathbb{Z}$:

$$\{ (0, k) \mid k \in n\mathbb{Z} \} \mbox{ for any } n \in \mathbb{N}$$
$$\{ (k, 0) \mid k \in n\mathbb{Z} \} \mbox{ for any } n \in \mathbb{N}$$
$$\{ (nk, nl) \mid n \in \mathbb{Z} \} \mbox{ for any } k, l \in
\mathbb{Z} $$

\subsection{Article 37. Congruence modulo a subgroup}

\paragraph{Problem 37$\alpha$.}
Show that all the left cosets of a group $G$ with respect to a subgroup $H$
have the same number of elements; in other words, show that any two left cosets
are in one-to-one correspondence.

\paragraph*{Solution}
We will show $f : xH \rightarrow yH$ defined by $f (xh) = yh$ is a one-to-one
correspondence. The mapping $f$ is one-to-one since $xh_1, xh_2 \in xH$ such that
$f(xh_1)= f(xh_2) $ implies $yh_1 = yh_2$, or $h_1 = h_2$. The mapping $f$ is onto
since for every element $yh \in yH$ there exists an element in $xH$, namely $xh$,
such that $f (xh) = yh$. Thus $f$ is a one-to-one correspondence.

\paragraph{Problem 37$\beta$.}
Let $H$ be a subgroup of a group $G$. Define an equivalence relation on $G$
which partitions $G$ into right cosets, that is, subsets of the form
$$Hy = \{ x \in G \mid x = h y, h \in H \}. $$
Prove that the number of right cosets is the same as the number of left cosets
of $G$ with respect to $H$; that is, show that the set of right cosets and the
set of left cosets are in one-to-one correspondence.

\paragraph*{Solution}
Define a mapping $f : \{ xH \} \rightarrow \{Hy\}$ from the set of left cosets
to the set of right cosets by $f (xH) = Hx^{-1}$. We will show that $f$ is a
one-to-one correspondence. Suppose $f(xH) = f(yH)$ for two left cosets. Then
$Hx^{-1} = Hy^{-1}$. This implies $hx^{-1} = y^{-1}$ for some $h \in H$ which
in turn implies $x = yh$ (since $yhx^{-1} = e$). Thus $xH = yH$ and $f$ is
one-to-one. Now suppose $f$ is not onto. Then there exists a $y$ such that
$f(xH) \neq Hy$ for all $x \in G$. Since $f(y^{-1}H) = Hy$ if $y^{-1}$ were in $G$,
this amounts to saying that $y$ has no inverse, a contradiction since $G$ is a
group. Thus $f$ is a one-to-one correspondence.

\paragraph{Problem 37$\gamma$.}
Show that when $G$ is abelian, every right coset is a left coset modulo $H$.

\paragraph*{Solution}
Pick a right coset $Hx$. Then every element in $Hx$ can be written $hx$ for
some $h \in H$. Since $G$ is abelian, $hx = xh$ and thus $Hx = xH$.

\paragraph{Problem 37$\delta$.}
Let an equivalence relation on $S_n$ be defined by $\pi ~ \tau$ if and only if
$\pi n = \tau n$. Show that this equivalence relation is congruence modulo a
subgroup of $S_n$.

\paragraph*{Solution}
$H = \{ \pi \in S_n \mid \pi n = n \}$.

\subsection{Article 38. Order of a group}

\paragraph{Problem 38$\alpha$.}

\paragraph*{Solution}
Suppose $HK$ is a subgroup of $G$. Pick $hk \in HK$. Since $HK$ is a subgroup,
we have $(hk)^{-1} = k^{-1}h^{-1} \in HK$ so $k^{-1}h{-1} = h'k'$ for some
$h' \in H$, $k' \in K$. But then $hk = ((hk)^{-1})^{-1} = (h'k')^{-1}$ and
we have $hk \in KH$.

Now pick $kh \in KH$. Since $K$ and $H$ are both subgroups, $k = k_1^{-1}$ for
some $k_1 \in K$, $h = h_1^{-1} $ for some $h_1 \in H$, and
$kh = k_1^{-1}h_1^{-1} = (h_1 k_1)^{-1} \in HK$ since $HK$ is a subgroup. This
shows that $HK = KH$ if $HK$ is a subgroup.

Now suppose that $HK = KH$. If $hk, h'k' \in HK$, then $(hk)(h'k') = h(kh')k'
= $ ??.
If $hk \in HK$, then $k^{-1}h^{-1} = (hk)^{-1} \in
HK = KH$.

\paragraph{Problem 38$\gamma$.}
Let $G$ be a nontrivial group with no proper subgroups except the trivial one.
Show that $G$ is finite and that the order of $G$ is prime.

\paragraph*{Solution}
Suppose $G$ is a nontrivial group, has no proper subgroups except the trivial
one, and is infinite. Pick $a \in G$, $a \neq e$. $H=\left\{ a^m \mid m \in
\mathbb{Z} \right\}$ is a subgroup of $G$ and we must have $H=G$. However,
$H' = \left\{ (a^2)^m \mid m \in \mathbb{Z}\right\}$ is also a subgroup and
$H' \neq G$, a contradiction. Hence, $G$ must be finite.

Now suppose that $o(G) = kn$ is not prime. Pick $a\in G$, $a\neq e$ and let
$H = \left\{ a^m \mid m \in \mathbb{Z} \right\}$. We must have $H=G$. But then
$\left<a^n\right> = \left\{ (a^n)^m \mid m \in \mathbb{Z} \right\}$ is a proper subgroup,
a contradiction. Thus the order of $G$ must be prime.

\subsection{Article 39. The index of a group}

\paragraph{Problem 39$\alpha$.}
Let $H$ denote the subgroup of $S_n$ consisting of all elements $\pi \in S_n$
such that $\pi n = n$. What is $[S_n:H]$?

\paragraph*{Solution}
Since $o(H) = (n-1)!$ and $o(S_n) = n!$ then $[S_n:H] = n! / (n-1)! = n$ (see
\textbf{40}, Lagrange's Theorem).

\subsection{Article 41. Order of an element}

\paragraph{Problem 41$\alpha$.}
Determine the order of $a^m$ where $a\in G, o(a)=n$.

\paragraph*{Solution}
We know $a^n=e$. Write $m=kn+r$; then $a^m=a^{kn+r}=a^{kn}a^r=a^r$.
$o(a^m)$ is the smallest $k \in \mathbb{N}$ such that $(a^m)^k=e$.
Answer is $n /(n, m)$.

\begin{center}
\begin{tabular}{|c|c|c|c|c|}
\hline
n & m & $o(a^m)$ & lcm(n,m) & gcd(n,m) \\
\hline
12 & 1 & 12 & 12 & 1\\
12 & 2 & 6 & 12 & 2 \\
12 & 3 & 4 & 12 & 3 \\
12 & 4 & 3 & 12 & 4 \\
12 & 5 & 12 & 60 & 1 \\
12 & 6 & 2 & 12 & 6 \\
12 & 7 & 12 & 84 & 1 \\
12 & 8 & 3 & 24 & 4 \\
12 & 9 & 4 & 36 & 3 \\
12 & 10 & 6 & 60 & 2 \\
12 & 11 & 12 & 132 & 1 \\
12 & 12 & 1 & 12 & 12 \\
\hline
\end{tabular}
\end{center}

\paragraph{Problem 41$\beta$.}
Let $a, b \in G$, $G$ abelian. Describe $o(ab)$ in terms of $o(a)$ and $o(b)$.

\paragraph*{Solution}
$o(ab)$ is the smallest number $n$ such that $(ab)^n=e$. Since $G$ is abelian
we have $e=(ab)^n=(ab)(ab)\cdots(ab)=(a\cdots a)(b\cdots b)=a^n b^n$.

\paragraph{Problem 41$\gamma$.}
Show that $o(axa^{-1}) = o(x)$ for any elements $a$ and $x$ of a group $G$.

\paragraph*{Solution}
This follows from the relation $(axa^{-1})^k = ax^k a^{-1}$.

\paragraph{Problem 41$\delta$.}
For any elements $a$ and $b$ of a group $G$, show that $o(ab) = o(ba)$.

\paragraph*{Solution}
If $o(ab) = k$ then $(ab)^k = e$. This implies $(ba)^k = a^{-1}(ab)^k a = a^{-1}a = e$.

\paragraph{Problem 41$\epsilon$.}
Let $G$ be an abelian group. Show that for any natural number $n$, the set $G_n
= \{ x \in G \mid o(x) | n \}$ is a subgroup of $G$.

\paragraph*{Solution}
If $x, y \in G_n$ then $o(x) | n$ and $o(y) | n$ imply $[o(x), o(y)] | n$ so
$xy \in G_n$. Since $\left<x\right> = \left<x^{-1}\right>$, we have $o(x) | n$ implies $o(x^{-1}) | n$
so $x^{-1} \in G_n$. Hence $G_n$ is a subgroup of $G$.

\paragraph{Problem 41$\zeta$.}
Prove there can only be two groups of order 4 (up to isomorphism). Do the same
for order 6.

\paragraph*{Solution}
Lagrange's theorem (see 40) implies that $o(a) | o(G)$. For $o(G)=4$, we must have
$o(a)=1$, $o(a)=2$, or $o(a)=4$. $o(a)=1$ implies $a=a^1=e$. $o(a)=4$ implies
$G$ is cyclic and $\left<a\right> = G$.

For $o(G)=6$, we must have $o(a)=1$, $o(a)=2$, $o(a)=3$ or $o(a)=6$.

\subsection{Article 44. Roots of unity}

\paragraph{Problem 44$\beta$.}
Let $\zeta$ be a primitive $n$-th root of unit and let $\xi$ be a primitive
$m$-th root of unity. What is the subgroup of the circle group $K$ generated
by $\zeta$ and $\xi$?

\paragraph*{Solution}
$e^{2\pi i / [m,n]}$ generates so $K_{[m,n]}$.

\paragraph{Problem 44$\gamma$.}
Describe the group $K_n \cap K_m$.

\paragraph*{Solution}
$K_{(m,n)}$.


\section{Conjugacy, normal subgroups, and quotient groups}


\subsection{Article 46. Normal Subgroups}

\paragraph{Problem 46$\alpha$.}
Show that a subgroup $H$ of a group $G$ is normal if and only if every left
coset of $H$ is equal to some right coset of $H$.

\paragraph*{Solution}
Suppose $H$ is a normal subgroup of $G$. Then $H = a^{-1}Ha$ for any $a \in G$
implies that $aH = Ha$ for any $a$ and therefore every left coset equals some
right coset.

Now suppose that every left coset equals a right coset. Pick $a \in G$. Then
there exists $b \in G$ such that $aH = Hb$. Since $a \in Ha$, we get $a \in Hb$.
So $Hb = Ha = aH$ and therefore $H = H^a$. Since $a$ was arbitrary, we have
$H$ is a normal subgroup.

\paragraph{Problem 46$\gamma$.}
Show that a subgroup of index 2 is always normal.

\paragraph*{Solution}
Let $H$ be a subgroup of index 2 of a group $G$. Then there are only two left
cosets and two right cosets. Since $H$ is both a left coset and right coset,
the other left coset is the same as the other right coset, namely $H^c$.

\subsection{Article 47.}

\paragraph{Problem 47$\alpha$.}
Describe the quotient group $\mathbb{Z} / m\mathbb{Z}$.

\paragraph*{Solution}
$[0]_m, [1]_m, \dots, [m-1]_m$.

\paragraph{Problem 47$\beta$.}
$\mathbb{Z}$ is a normal subgroup of $\mathbb{R}$ (real numbers under addition).
Show that the quotient group $\mathbb{R/Z}$ is isomorphic to the circle group
$K$ (\textbf{31}).

\paragraph*{Solution}
We have $x, y \in \mathbb{R}$ are in the same coset $x + \mathbb{Z}$ if
$y = x + n$ for some $n \in \mathbb{Z}$. Since every $ y \in \mathbb{R} $ can
be written $ y = x + n$ for some $ 0 \leq x < 1 $ and $ n \in \mathbb{Z}$, the
quotient group $\mathbb{R/Z} = \{ x + \mathbb{Z} \mid 0 \leq x < 1 \}$.
Define $\phi : \mathbb{R/Z} \rightarrow K$ as
$\phi (x+\mathbb{Z}) = e^{2\pi i x}$. $\phi$ is onto, and since $ 0 \leq x < 1$,
$\phi$ is also one-to-one.

To show that $\mathbb{R/Z}$ and $K$ are isomorphic, we need to show that $\phi$
preserves group products. We have
$ \phi ((x+\mathbb{Z})(y+\mathbb{Z})) = \phi ((x+y)+\mathbb{Z}) = e^{2\pi i (x+y)}
= e^{2\pi i x}e^{2 \pi i y} = \phi (x+\mathbb{Z}) \phi (y+\mathbb{Z})$ [here it
is understood that $x+y$ "wraps" around the $[0,1)$ interval as needed].

\subsection{Article 48. Normalizer of a subset}

\paragraph{Problem 48$\gamma$.}
Show that $N(S^a) = N(S)^a$ for any subset $S$ of a group $G$.

\paragraph*{Solution}
$x \in N(S^a)$ if and only if $S^{ax} = S^a$ if and only if $S^{axa^{-1}} = S$
if and only if $x \in N(S)^a$.

($N(S^a) = \left\{ x \in G \mid S^{ax} = S^a \right\}$,
$N(S)^a = \left\{ x \in G \mid axa^{-1} \in N(s) \right\}$)

\subsection{Article 49.}

\paragraph{Problem 49$\alpha$.}
Let $S$ be a subset of a group $G$, that has exactly two conjugates. Show that
$G$ has a proper nontrivial normal subgroup.

\paragraph*{Solution}
By the theorem in \textbf{49}, $[G : N(S)] = 2$. By \textbf{46$\gamma$}, $N(S)$
must be a normal subgroup of $G$. Since the index of $N(S)$ in $G$ is 2, $N(S)$ is a
proper subgroup of $G$. To see that $N(S)$ is nontrivial, we only need to show
that $o(G) > 2$ since $o(N(S)) = o(G)/2$ by Lagrange's theorem (\textbf{40}).
$o(G) \neq 1$ because then there could not be exactly two conjugates. If we
consider a group of order 2, we find that each element is only conjugate to
itself, hence $o(G) \neq 2$.

\paragraph{Problem 49$\beta$.}
Let $H$ be a proper subgroup of a finite group $G$. Show there is at least
one element of $G$ not contained in $H$ or any of its conjugates.

\paragraph*{Solution}

\subsection{Article 50. Center of a group}

\paragraph{Problem 50$\alpha$.}
Show that the center of the symmetric group $S_n$ is trivial for $n > 2$.

\paragraph*{Solution}
See \textbf{79$\gamma$}.

\subsection{Article 51. Conjugacy class equation}

\paragraph{Problem 51$\alpha$.}
Divide the elements of the quaternion group $Q$ into conjugacy classes and
verify the conjugacy class equation.

\paragraph*{Solution}
$Z_Q = \{ e, a^2 \}$, $C_a = \{ a, a^3 \}$, $C_b = \{ b, a^2 b \}$,
$C_{ab} = \{ ab, a^3b \}$.


\section{The Sylow theorems}

\subsection{Article 54. Orbits and Stabilizers}
\paragraph{Problem 54$\alpha$.}
Let $H$ be a subgroup of a group $G$. Then $H$ acts on $G$ by the rule
$h * x = hx$ for $h \in H$, $x \in G$. What is the orbit of $x \in G$ under
this action? What is the stabilizer $H_x$?

\paragraph*{Solution}
The orbit of $x$ is the set $H * x = \{ y \in G \mid y = h * x \mbox{ for some }
h \in H \} $. Since $ h * x = hx$, this is just the right coset $Hx$. The
stabilizer is the set $ H_x = \{ h \in H \mid h*x = x \}$. Since $h*x=hx$ and
the identity is unique, the only element in $H$ such that $hx =x$ is $e$, so
$H_x = \{ e \}$.

\paragraph{Problem 54$\beta$.}

\paragraph*{Solution}
Since $x$ and $y$ are in the same orbit, we can pick $h \in G$ such that
$ y = h*x $.


\section{Group homomorphism and isomorphism}

\subsection{Article 60. Group homomorphism}

\paragraph{Problem 60$\alpha$.}
Show that a group homomorphism preserves identity elements and inverses.

\paragraph*{Solution}
Suppose $\phi$ is a group homomorphism. Then $\phi e = \phi (ee) = (\phi e) (\phi e)$
which implies $\phi e = e'$. We then have $e' = \phi e = \phi (aa^{-1}) =
(\phi a) (\phi a^{-1})$ which implies $\phi( a^{-1}) = (\phi a)^{-1}$.

\paragraph{Problem 60$\beta$.}
Let $H$ be a normal subgroup of a group $G$. Show that the mapping $\phi : G
\rightarrow G/H$ given by $\phi g = gH$ is a group homomorphism.

\paragraph*{Solution}
We have $\phi (ab) = (ab)H = (aH)(bH) = (\phi a)(\phi b)$ (see \textbf{47}).

\paragraph{Problem 60$\gamma$.}
Show that a group $G$ is abelian if and only if the mapping $\phi : G \rightarrow
G$ given by $\phi g = g^{-1}$ is an endomorphism of $G$.

\paragraph*{Solution}
If $\phi$ is an endomorphism, we have $ab = \phi(b^{-1}a^{-1}) = \phi(b^{-1})\phi(a^{-1})
= ba$ so $G$ is abelian. On the other hand, if $G$ is abelian, then $\phi (ab)
= b^{-1}a^{-1} = a^{-1}b^{-1} = (\phi a)(\phi b)$ so that $\phi$ is an endomorphism.

\paragraph{Problem 60$\delta$.}
Show that a group $G$ is abelian if and only if the mapping $\phi : G \rightarrow
G$ given by $\phi g = g^2$ is an endomorphism of $G$.

\paragraph*{Solution}
If $\phi$ is an endomorphism, then $abab = \phi(ab) = \phi a \phi b = aabb$ which
implies $ab = ba$ and hence $G$ is abelian. On the other hand, suppose $G$ is
abelian. Then $\phi (ab) = abab = aabb = (\phi a)(\phi b)$ which shows that $\phi$
is an endomorphism.

\paragraph{Problem 60$\epsilon$.}
Show that a group $G$ is abelian if and only if the mapping $\phi : G \times G \rightarrow
G$ given by $\phi (a,b) = ab$ is a group homomorphism.

\paragraph*{Solution}
If $\phi$ is a group homomorphism, then $acbd = \phi(ac,bd) = \phi((a,b)(c,d))
= \phi(a,b)\phi(c,d) = abcd$ which implies $bc = cb$ for any elements $c,b \in G$
so $G$ must be abelian. Now suppose $G$ is abelian. Then $\phi ((a,b)(c,d)) =
\phi (ac,bd) = acbd = abcd = \phi(a,b)\phi(c,d)$, showing that $\phi$ must be
a group homomorphism.


\subsection{Article 62. Group isomorphism}

\paragraph{Problem 62$\alpha$.}
Let $\mathbb{R}$ denote the group of all real numbers under addition and let
$\mathbb{R}^+$ denote the group of all positive real numbers under multiplication.
Show that the mapping $\phi : \mathbb{R} \rightarrow \mathbb{R}^+$ given by
$\phi x = e^x$ is an isomorphism. What is the inverse of $\phi$?

\paragraph*{Solution}
$\phi$ is a one-to-one correspondence so we only need show that it preserves group products:
$\phi(x+y) = e^{x+y} = e^x e^y = (\phi x)(\phi y)$. The inverse $\phi^{-1}: \mathbb{R}^+
\rightarrow \mathbb{R}$ is given by $\phi^{-1} x = \ln x$.

\paragraph{Problem 62$\beta$.}
Let $\phi : \mathbb{Z}_{16} \rightarrow \mathbb{Z'}_{17}$ be given by
$\phi [k]_{16} = [3^k]_{17}$. Show that $\phi$ is an isomorphism.

\paragraph*{Solution}
$\phi$ is a homomorphism because $\phi([k]_{16} + [l]_{16}) = \phi([k+l]_{16}) = [3^{k+l}]_{17}
= [3^k]_{17}[3^l]_{17} = (\phi [k]_{16})(\phi[l]_{16})$. To see that $\phi$ is an isomorphism,
we must show that $\phi$ is a one-to-one correspondence. ??

\paragraph{Problem 62$\epsilon$.}
Let $G$ denote the group of real numbers between -1 and 1 under the operation
$x y = (x+y)/(1+xy)$. Show that $G$ is isomorphic to the group of real numbers
$\mathbb{R}$ under addition.

\paragraph*{Solution}
Consider $\phi x = (1-e^{-x}) / (1 + e^{-x})$. $\phi$ is a one-to-one correspondence
$\phi : \mathbb{R} \rightarrow G$. $\phi$ also preserves group products:
\begin{eqnarray*}
\phi(x)\phi(y) & = & \frac{(\frac{1-e^{-x}}{1+e^{-x}}) + (\frac{1-e^{-y}}{1+e^{-y}})}
{1 + (\frac{1-e^{-x}}{1+e^{-x}})(\frac{1-e^{-y}}{1+e^{-y}})} \\
& = & \frac{(1-e^{-x})(1+e^{-y}) + (1-e^{-y})(1+e^{-x})}
{(1+e^{-x})(1+e^{-y})+(1-e^{-x})(1-e^{-y})} \\
& = & \frac{2(1-e^{-x}e^{-y})}{2(1 + e^{-x}e^{-y})} \\
& = & \frac{1 - e^{-(x+y)}}{1+e^{-(x+y)}} \\
& = & \phi (x+y)
\end{eqnarray*}
Hence $\phi$ is an isomorphism and $\mathbb{R}$ and $G$ are isomorphic.

\subsection{Article 64.}

\paragraph{Problem 64$\alpha$.}
Show that the set $\mathcal{A}(G)$ of all automorphisms of a group $G$
is a group under composition and that the set $\mathcal{I}(G)$ of inner
automorphisms of a group $G$ is a normal subgroup of $\mathcal{A}(G)$.

\paragraph*{Solution}
By \textbf{16$\alpha$}, $\mathcal{A}(G)$ is a group.

Pick $\phi \in \mathcal{A}(G)$. For any $\alpha_a \in \mathcal{I}(G)$,
we have
$$ (\phi^{-1} \alpha_a \phi)(g) = \phi^{-1}(a^{-1}\phi(g)a)
= \phi^{-1}(a)^{-1} g \phi^{-1}(a) $$
so that $\phi^{-1} \alpha_a \phi = \alpha_{\phi^{-1}(a)} \in \mathcal{I}(G)$.
This shows $\mathcal{I}(G) \subset \mathcal{I}(G)^\phi$.

\subsection{Article 67. Factoring homomorphisms}

\paragraph{Problem 67$\alpha$.}
Let $\mathbb{R}$ denote the group of real numbers under addition and
$\mathbb{C}^*$ the group of nonzero complex numbers under multiplication.
Decompose the homomorphism $ \phi : \mathbb{R} \rightarrow \mathbb{C}^*$
given by $\phi x = e^{2 \pi i x}$ in the manner of the proposition above.

\paragraph*{Solution}
We have $\mathbb{R} / \mbox{Ker} \phi = \mathbb{Z}$. $\gamma : \mathbb{R} \rightarrow
\mathbb{R} / \mbox{Ker } \phi $ can be given by $ \gamma x = x - [x] $. We also
have $ \mbox{Im } \phi = \left\{ z \mid |z| = 1\right\}$, the unit circle. Then
$\beta : \mathbb{R} / \mbox{Ker } \phi \rightarrow \mbox{Im } \phi $ can be
given by $ \beta (\gamma x) = e^{2\pi i (\gamma x)} = \phi (\gamma x)$.
Lastly, $\alpha : \mbox{Im } \phi \rightarrow \mathbb{C}^* $ is given by
$ \alpha ( \beta ( \gamma x ) ) = \beta (\gamma x) $.

\section{Normal and composition series}

\subsection{Article 74.}

\paragraph{Problem 74$\beta$.}
Construct composition series for the groups $\mathbb{Z}_8$, $D_4$, and $Q$.

\paragraph*{Solution}
For $\mathbb{Z}_8$: $\{e\} = \{[0]_8\}$, $\{[0]_8, [4]_8\}$, $\{[0]_8, [2]_8, [4]_8, [6]_8\}$,
$\{[0]_8, [1]_8, [2]_8, [3]_8, [4]_8, [5]_8, [6]_8, [7]_8\}$ = $\mathbb{Z}_8$.

For $D_4$: $\{e\}$, $\{e, \alpha^2\}$, $\{e, \alpha, \alpha^2, \alpha^3\}$, $D_4$.

For $Q$: $\{e\}$, $\{e, a^2\}$, $\{e, a, a^2, a^3\}$, $Q$.


\section{The Symmetric groups}


\subsection{Article 79. Cyclic permutations}

\paragraph{Problem 79$\alpha$.}
Compute the number of distinct $k$-cycles in $S_n$.

\paragraph*{Solution}
There are $\binom{n}{k} = \frac{n!}{k!(n-k)!}$
ways to choose $k$ elements from $S_n$. To avoid identical cycles written in
different ways, fix the first element of the cycle as the smallest element among
the $a_k$. Then there are $(k-1)!$ ways to arrange the $a_k$ to form distinct
cycles. Putting this together, we get $ \frac{n!}{k(n-k)!}$ $k$-cycles.

\paragraph{Problem 79$\gamma$.}
Prove that disjoint cyclic permutations commute.

\paragraph*{Solution}
Suppose $\pi = (a_1, a_2, \dots, a_k)$ and $\tau = (b_1, b_2, \dots, b_l)$ and
$\{ a_1, a_2, \dots, a_k \} \cap \{ b_1, b_2, \dots, b_l \} = \emptyset$.
For $ i \in \{ a_1, a_2, \dots, a_k \} $, $(\pi \tau)(i) = \tau(\pi (i)) =
\pi(i) = \pi(\tau(i)) = (\tau \pi)(i)$. A similar relation holds for $ i \in
\{ b_1, b_2, \dots, b_l \}$. For $i \in N_n - \{ a_1, a_2, \dots, a_k \} \cup
\{ b_1, b_2, \dots, b_l \}$ we have $(\pi \tau)(i) = i = (\tau \pi)(i)$. Thus
$\pi \tau = \tau \pi$.

\paragraph{Problem 79$\delta$.}
Show that $S_n$ contains $\binom{n}{k}$
subgroups isomorphic to $S_k \times S_{n-k}$, all of which are conjugates.

\paragraph*{Solution}
Let $a_1, a_2, \dots , a_k \in \mathbb{N}_n$ be distinct integers. There are
$n-k$ integers in $\mathbb{N}_n - \left\{a_1, \dots , a_k\right\}$; call these
$b_1, b_2, \dots , b_{n-k}$. There are $\binom{n}{k}$ ways to make this selection
and each choice gives rise to a subgroup. Each element in the subgroup can be
written as
$$
h_{\pi, \sigma} =
\left(
\begin{array}{cccccccc}
a_1        & a_2        & \dots & a_k        & b_1           & b_2           & \dots & b_{n-k} \\
a_{\pi(1)} & a_{\pi(2)} & \dots & a_{\pi(k)} & b_{\sigma(1)} & b_{\sigma(2)} & \dots & b_{\sigma(n-k)}
\end{array}
\right)
$$
for some $(\pi, \sigma) \in S_k \times S_{n-k}$

\subsection{Article 80.}

\paragraph{Problem 80$\gamma$.}
Show that $S_n$ is generated by the transpositions $$(1,2), (2,3), \dots, (n-1,n)$$.

\paragraph*{Solution}
$(1,2)(2,3)(1,2) = (1,3)$.
$(1,2)(2,3)(3,4)(2,3)(1,2) = (1,4)$.
In general, $$(1,2)(2,3) \cdots (k-2,k-1)(k-1,k)(k-2,k-1) \cdots (2,3)(1,2) = (1,k).$$
In general, for $l < k$,
$$(l,l+1)(l+1,l+2) \cdots (k-2,k-1)(k-1,k)(k-2,k-1) \cdots (l+1,l+2)(l,l+1) = (l,k).$$
So $(1,2), (2,3), \dots, (n-1,n)$ can generate any tranposition. By the corollary
in \textbf{80}, this shows that $(1,2), (2,3), \dots, (n-1,n)$ generates $S_n$.

\paragraph{Problem 80$\delta$.}
Show that $S_n$ is generated by the cycles $(1,2)$ and $(1,2, \dots, n)$.

\paragraph*{Solution}
We only need to show that $(1,2)$ and $(1,2, \dots, n)$ generate
$(1,2), (2,3), \dots, (n-1,n)$, then apply \textbf{80$\gamma$}.

Let $\pi = (1,2,\dots,n)$ and $\tau =(1,2)$. Note that $\pi$ is cyclic of order
$n$ so $\pi^{-1} = \pi^{n-1}$, $\pi^{-2} = \pi^{n-2}$, etc.
By direct calculation, $\pi^{-1}\tau\pi = (2,3)$. We can also see that
$\pi^{-2}\tau\pi^2 = (3,4)$. In general, $$\pi^{-k}\tau\pi^k = (k+1,k+2)$$ for
$k \in \mathbb{N}_{n-2}$. This shows that $(1,2)$ and $(1,2, \dots, n)$ generate
$(1,2), (2,3), \dots, (n-1,n)$.

\subsection{Article 81. Even and Odd Permutations}

\paragraph{Problem 81$\alpha$.}
Determine the sign of a $k$-cycle in terms of $k$.

\paragraph*{Solution}
$(-1)^{k+1}$. 2-cycles are odd, 3-cycles are even, 4-cycles are odd, etc.

