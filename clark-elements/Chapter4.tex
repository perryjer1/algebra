\chapter{Galois Theory}

\section{Automorphisms}


\subsection{Article 122.}

\paragraph{Problem 122$\gamma$.}
Determine the group of automorphisms of $\mathbb{Q}(i)$ and $\mathbb{Q}(\sqrt{2})$.

\paragraph*{Solution}
We know $[\mathbb{Q}(i) : \mathbb{Q}] = 2$ since a minimal polynomial for $i$ over
$\mathbb{Q}$ is $x^2 + 1$. Combined with \textbf{125}, we only need look for
two automorphisms. The first is the identity, $\phi_1 (a + bi) = a+bi$. The
second is $\phi_2 (a+bi) = a- bi$.

Similarly, $[\mathbb{Q}(\sqrt{2}) : \mathbb{Q}] = 2$ since $x^2 - 2$ is a minimal
polynomial for $\sqrt{2}$ over $\mathbb{Q}$. The two automorphisms are
$\phi_1 (a+b\sqrt{2}) = a + b\sqrt{2}$ and $\phi_2 (a +b\sqrt{2}) = a-b\sqrt{2}$.

\paragraph{Problem 122$\epsilon$.}
Let $\phi$ be an automorphism of a field $E$. Prove that the set
$$ F = \{ \alpha \in E \mid \phi \alpha = \alpha \} $$
is a subfield of $E$.

\paragraph*{Solution}
We need to show $F$ is a subgroup of $E$ under addition, and $F^*$ is a subgroup
of $E^*$ under multiplication.

For $\alpha, \beta \in F$, we have
$\phi (\alpha + \beta) = \phi(\alpha) + \phi(\beta) = \alpha + \beta$ so that
$\alpha + \beta \in F$. For $\alpha \in F$, $0 = \phi(0) = \phi (\alpha + (-\alpha)) =
\phi(\alpha) + \phi(-\alpha) = \alpha + \phi(-\alpha)$ which implies
$\phi(-\alpha) = -\alpha$; hence $-\alpha \in F$. This shows that $F$ is a
subgroup of $E$ under addition.

For $\alpha, \beta \in F^*$, we have
$\phi (\alpha \beta) = \phi(\alpha) \phi(\beta) = \alpha \beta$ so that
$\alpha \beta \in F^*$. For $\alpha \in F$, $1 = \phi(1) = \phi (\alpha \alpha^{-1}) =
\phi(\alpha) \phi(\alpha^{-1}) = \alpha \phi(\alpha^{-1})$ which implies
$\phi(\alpha^{-1}) = \alpha^{-1}$; hence $\alpha^{-1} \in F^*$. This shows that $F^*$ is a
subgroup of $E^*$ under multiplication.

\subsection{Article 123. Fixed elements, fixed fields}

\paragraph{Problem 123$\alpha$.}
Let $\zeta = e^{2\pi i/5}$, and let $\phi$ denote the automorphism of
$\mathbb{Q}(\zeta)$ given by $\phi \zeta = \zeta^4$. Prove that the fixed
field of $\phi$ is $\mathbb{Q}(\sqrt{5})$.

\paragraph*{Solution}
From $110$, we know that $\left\{1, \zeta, \zeta^2, \zeta^3 \right\}$
is a basis for $\mathbb{Q}(\zeta)$; in other words, every element in $\mathbb{Q}(\zeta)$
can be written $a + b\zeta + c\zeta^2 + d\zeta^3$ for $a, b,c,d \in \mathbb{Q}$.
The automorphism $\phi$ must have $\phi\zeta^2 = \phi\zeta \phi\zeta = \zeta^4 \zeta^4
= \zeta^8 = \zeta^3$, $\phi\zeta^3 = \phi\zeta^2 \phi\zeta = \zeta^3 \zeta^4
= \zeta^7 = \zeta^2$, and $\phi\zeta^4 = \phi\zeta^2 \phi\zeta^2 = \zeta^3 \zeta^3
= \zeta^6 = \zeta$. We also have $\phi(1) = 1$ and $\phi(0) = 0$, therefore
$\phi x = x$ for any $x \in \mathbb{Q}$. Putting this all together gives the
automorphism as $\phi(a+b\zeta+c\zeta^2+d\zeta^3)=a+b\zeta^4+c\zeta^3+d\zeta^2$
for $a,b,c,d \in \mathbb{Q}$. To leave an element fixed, we need $b=0$ and $c=d$,
so and element in the fixed field has the form $a + b(\zeta^2 +\zeta^3) =
a + b(\cos(4\pi/5)+\cos(6\pi/5)) = a - 2b\cos(\pi/5) = a - 2b(1+\sqrt{5})/4 =
(a-b/2) - (b\sqrt{5}/2)$. Writing $a' = a - b/2$ and $b' = -b/2$, we see that this
is equivalent to $a' + b'\sqrt{5}$ for $a',b' \in \mathbb{Q}$. This shows that
the fixed field is in fact $\mathbb{Q}(\sqrt{5})$.

\paragraph{Problem 123$\beta$.}
Let $\phi$ be an automorphism of a field $E$ leaving fixed the subfield $F$.
Show that $\alpha \in E$ and a root of $f \in F[x]$ implies $\phi\alpha$ is also
a root of $f$.

\paragraph*{Solution}
By hypothesis, we have $f\alpha = 0$. We can write $fx$ as
$\sum_{k=0}^n a_k x^k$, where $a_k \in F$. Then we have
$0 = \phi(0) = \phi(f\alpha) = \phi(\sum_{k=0}^n a_k \alpha^k)
= \sum_{k=0}^n \phi(a_k) \phi(\alpha^k) = \sum_{k=0}^n a_k (\phi\alpha)^k
= f(\phi\alpha)$.

\paragraph{Problem 123$\gamma$.}
Let $\phi$ be an automorphism of a field $E$ with fixed field $F$. Show that
$\phi$ extends uniquely to a mapping $$\hat{\phi} : E[x] \rightarrow E[x]$$ with the
following properties:
\begin{enumerate}
\item $\hat{\phi}(c) = \phi(c)$ for any constant polynomial $c$,
\item $\hat{\phi}(x) = x$,
\item $\hat{\phi}(f+g) = \hat{\phi}(f) + \hat{\phi}(g)$,
\item $\hat{\phi}(fg) = (\hat{\phi} f)(\hat{\phi} g)$.

\end{enumerate}
Furthermore, show that $\hat{\phi}(f) = f$ if and only if $f \in F[x]$.

\paragraph*{Solution}
Define $\hat{\phi} : E[x] \rightarrow E[x]$ as follows:
$$ \hat{\phi}\left(\sum_{k=0}^{n} a_k x^k\right) = \sum_{k=0}^{n} \phi(a_k)x^k$$
Property 1 is obvious.
Property 2 follows from $\phi(1) = 1$ (since $\phi$ is an automorphism).
Properties 3 and 4 follow from $\phi$ being an automorphism and the definition
of $\hat{\phi}$.

If $f \in F[x]$, then $\phi(a_k) = a_k$ and hence $\hat{\phi}(f) = f$. On the
other hand, suppose $f \in F[x]$ so that $\hat{\phi}(f) = f$. This means we
must have $\phi(a_k) = a_k$ for each coefficient $a_k$. In other words,
$a_k \in F$ for each $k$ or what is the same thing, $f \in F[x]$.

To show uniqueness, suppose we have two mappings $\hat{\phi}_1$ and $\hat{\phi}_2$
that satisfy the properties.
Using 2 and 4 we can show inductively that $\hat{\phi} (x^n) = x^n$. Combining
this with 1 and 4 gives $\hat{\phi}(c x^n) = \phi(c) x^n$. These show that
$\hat{\phi}_1(c x^n) = \hat{\phi}_2(c x^n)$ for all $n$ and therefore
$\hat{\phi}_1 = \hat{\phi}_2$.

\paragraph{Problem 123$\epsilon$.}
Let $E$ be the splitting field in $\mathbb{C}$ of the polynomial $x^4 + 1$.
Find automorphisms of $E$ which have fixed fields $\mathbb{Q}(\sqrt{-2})$,
$\mathbb{Q}(\sqrt{2})$, and $\mathbb{Q}(i)$. Is there an automorphism
with fixed field $\mathbb{Q}$?

\paragraph*{Solution}
Let $\xi = e^{\pi i / 4}$. Define automorphisms by
$\phi_0 \xi = \xi$,
$\phi_1 \xi = \xi^3$, $\phi_2 \xi = \xi^5$, and
$\phi_3 \xi = \xi^7$.

The fixed field of $\phi_2$ is $\mathbb{Q}(i)$. The fixed field of $\phi_3$
is $\mathbb{Q}(\sqrt{-2})$.

Since there are only four automorphisms (the identity being the fourth),
there are no automorphisms with fixed field $\mathbb{Q}$.

\subsection{Article 126. Fixed fields of finite groups of
  automorphisms}

\paragraph{Problem 126$\beta$.}
Find a group of automorphisms of $\mathbb{Q}(\zeta)$, where $\zeta = e^{2\pi i /5}$,
of which the fixed field is $\mathbb{Q}$, and determine
$\left[\mathbb{Q}(\zeta):\mathbb{Q}\right]$. How else can
$\left[\mathbb{Q}(\zeta):\mathbb{Q}\right]$ be found?

\paragraph*{Solution}
A group $G$ of automorphisms is
\begin{eqnarray}
1_E(a+b\zeta+c\zeta^2+d\zeta^3) &=& a+b\zeta+c\zeta^2+d\zeta^3 \nonumber \\
\phi(a+b\zeta+c\zeta^2+d\zeta^3) &=& a+b\zeta^2+c\zeta^4+d\zeta \nonumber \\
\phi^2(a+b\zeta+c\zeta^2+d\zeta^3) &=& a+b\zeta^4+c\zeta^3+d\zeta^2 \nonumber \\
\phi^3(a+b\zeta+c\zeta^2+d\zeta^3) &=& a+b\zeta^3+c\zeta+d\zeta^4 \nonumber
\end{eqnarray}
Their fixed field is $\mathbb{Q}$, so by 126, we have
$\left[\mathbb{Q}(\zeta):\mathbb{Q}\right] = o(G) = 4$. Note this group $G$ is
isomorphic to $Z'_5$ with isomorphism $\psi$ given by $\psi(1_E) = [1]_5$,
$\psi(\phi) = [2]_5$, $\psi(\phi^2) = [4]_5$, and $\psi(\phi^3) = [3]_5$
(see 122$\delta$).


\section{Galois extensions}

\subsection{Article 131.}

\paragraph{Problem 131$\alpha$.}
Express the following symmetric polynomials in terms of elementary
symmetric functions.

\paragraph*{Solution}

$$x_1^2 + x_1 x_2 + x_2^2 + x_2 x_3 + x_3^2 + x_3 x_1 = \sigma_1^2 - \sigma_2$$
$$x_1^2 x_2^2 + x_2^2 x_3^2 + x_3^2 x_1^2 = \sigma_2^2 - 2\sigma_3$$
$$(x_1-x_2)^2 (x_2-x_3)^2 (x_3-x_1)^2 = ??$$
$$x_1^3 + x_2^3 + x_3^3 = \sigma_1^3 - 3(\sigma_1 \sigma_2 - \sigma_3)$$
$$x_1^4 + x_2^4 + x_3^4 = ??$$


\section{Solvability of equations by radicals}

